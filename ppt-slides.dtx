% \iffalse meta-comment
% (The MIT License)
%
% Copyright (c) 2021-2022 Yegor Bugayenko
%
% Permission is hereby granted, free of charge, to any person obtaining a copy
% of this software and associated documentation files (the 'Software'), to deal
% in the Software without restriction, including without limitation the rights
% to use, copy, modify, merge, publish, distribute, sublicense, and/or sell
% copies of the Software, and to permit persons to whom the Software is
% furnished to do so, subject to the following conditions:
%
% The above copyright notice and this permission notice shall be included in all
% copies or substantial portions of the Software.
%
% THE SOFTWARE IS PROVIDED 'AS IS', WITHOUT WARRANTY OF ANY KIND, EXPRESS OR
% IMPLIED, INCLUDING BUT NOT LIMITED TO THE WARRANTIES OF MERCHANTABILITY,
% FITNESS FOR A PARTICULAR PURPOSE AND NONINFRINGEMENT. IN NO EVENT SHALL THE
% AUTHORS OR COPYRIGHT HOLDERS BE LIABLE FOR ANY CLAIM, DAMAGES OR OTHER
% LIABILITY, WHETHER IN AN ACTION OF CONTRACT, TORT OR OTHERWISE, ARISING FROM,
% OUT OF OR IN CONNECTION WITH THE SOFTWARE OR THE USE OR OTHER DEALINGS IN THE
% SOFTWARE.
% \fi

% \CheckSum{0}
%
% \CharacterTable
%  {Upper-case    \A\B\C\D\E\F\G\H\I\J\K\L\M\N\O\P\Q\R\S\T\U\V\W\X\Y\Z
%   Lower-case    \a\b\c\d\e\f\g\h\i\j\k\l\m\n\o\p\q\r\s\t\u\v\w\x\y\z
%   Digits        \0\1\2\3\4\5\6\7\8\9
%   Exclamation   \!     Double quote  \"     Hash (number) \#
%   Dollar        \$     Percent       \%     Ampersand     \&
%   Acute accent  \'     Left paren    \(     Right paren   \)
%   Asterisk      \*     Plus          \+     Comma         \,
%   Minus         \-     Point         \.     Solidus       \/
%   Colon         \:     Semicolon     \;     Less than     \<
%   Equals        \=     Greater than  \>     Question mark \?
%   Commercial at \@     Left bracket  \[     Backslash     \\
%   Right bracket \]     Circumflex    \^     Underscore    \_
%   Grave accent  \`     Left brace    \{     Vertical bar  \|
%   Right brace   \}     Tilde         \~}

% \GetFileInfo{ppt-slides.dtx}
% \DoNotIndex{\endgroup,\begingroup,\let,\else,\s,\n,\r,\\,\1,\fi}

% \iffalse
%<*driver>
\ProvidesFile{ppt-slides.dtx}
%</driver>
%<package>\NeedsTeXFormat{LaTeX2e}
%<package>\ProvidesPackage{ppt-slides}
%<*package>
[0000-00-00 0.0.0 Slide Decks a la Power Point]
%</package>
%<*driver>
\documentclass{ltxdoc}
\usepackage[T1]{fontenc}
\usepackage[tt=false, type1=true]{libertine}
\usepackage{microtype}
\AddToHook{env/verbatim/begin}{\microtypesetup{protrusion=false}}
\usepackage{graphicx}
\usepackage{xcolor}
\usepackage[runs=2,dtx,margin=0,small]{docshots}
\usepackage{href-ul}
\PageIndex
\EnableCrossrefs
\CodelineIndex
\RecordChanges
\begin{document}
  \DocInput{ppt-slides.dtx}
  \PrintChanges
  \PrintIndex
\end{document}
%</driver>
% \fi

% \title{|ppt-slides|: \LaTeX{} Package for \\ Slide Decks \`{a} la Power Point\thanks{The sources are in GitHub at \href{https://github.com/yegor256/ppt-slides}{yegor256/ppt-slides}}}
% \author{Yegor Bugayenko \\ \texttt{yegor256@gmail.com}}
% \date{\filedate, \fileversion}
%
% \maketitle
%
% \section{Introduction}
%
% This package helps you render slide decks in PowerPoint\texttrademark{} style:
% \begin{docshot}
% \documentclass{article}
% \usepackage[template,scheme=dark]{ppt-slides}
% \begin{document}
% \begin{pptMiddle}
% \pptTitle{Hello, world!}{How are you?}
% \end{pptMiddle}
% \end{document}
% \end{docshot}

% It is recommended to use this package together with
% \href{https://github.com/yegor256/clicks}{clicks}:
% \begin{docshot}
% \documentclass{article}
% \usepackage[template,scheme=dark]{ppt-slides}
% \usepackage[static]{clicks}
% \begin{document}
% We help you make slide decks in \LaTeX:
% \plick{\pptBanner{Making Slides Is Easy}}
% \plick{Just use this package...}
% \plush{together with 'clicks' package.}
% They will look similar to what PowerPoint can
% make, but with the precision of \LaTeX. We
% recommend using us together with ``clicks.''
% \end{document}
% \end{docshot}

% \section{Layout}

% \DescribeMacro{\pptToc}
% \DescribeMacro{\pptChapter}
% First, split your story into chapters:
%\iffalse
%<*verb>
%\fi
\begin{verbatim}
\documentclass{article}
\usepackage{clicks}
\usepackage[template,scheme=light]{ppt-slides}
\begin{document}
\pptToc
\plush{\pptChapter{About Me}}
...
\plush{\pptChapter[Idea]{My Idea Is Novel}}
...
\plush{\pptChapter[FAQ]{Discussion \& Questions}}
...
\end{document}
\end{verbatim}
%\iffalse
%</verb>
%\fi
% |\pptToc| will render the table of contents in an interactive ``clickable'' format.
% Thanks to the use of \href{https://github.com/yegor256/crumbs}{crumbs}, there will
% be a navigation at the top left corner.

% \DescribeMacro{\pptSection}
% Then, put sections inside chapters:
%\iffalse
%<*verb>
%\fi
\begin{verbatim}
\begin{document}
\pptToc
\plush{\pptChapter{About Me}}
\plush{\pptSection[Student]{I'm a Student}}
...
\plush{\pptSection[Athlete]{Also, I'm an Athlete}}
...
\plush{\pptChapter[Idea]{My Idea Is Novel}}
\plush{\pptSection{Novelty}}
\plush{\pptSection{Impact}}
\end{document}
\end{verbatim}
%\iffalse
%</verb>
%\fi
% The |\pptChapter| and |\pptSection| commands used together will render
% nice two-level nagivation menu at the top left corner.

% \DescribeMacro{\pptLeft}
% \DescribeMacro{\pptRight}
% You can place the title of the presentation and your name at the bottom
% of each slide, on the left and on the right respectively:
%\iffalse
%<*verb>
%\fi
\begin{verbatim}
\documentclass{article}
\usepackage{clicks}
\usepackage[template,scheme=light]{ppt-slides}
\pptLeft{How Did I Spend Last Summer}
\pptRight{Yegor Bugayenko}
\begin{document}
...
\end{document}
\end{verbatim}
%\iffalse
%</verb>
%\fi
% If you don't use the |\pptLeft| or |\pptRight| commands, nothing will
% be printed at the bottom.

% \section{Commands}

% \DescribeMacro{\pptPic}
% You can add an image to the slide (the first argument is the width
% of the image in relation to |\textwidth|,
% while the second one is the path of it):
% \begin{docshot}
% \documentclass{article}
% \usepackage[template,scheme=dark]{ppt-slides}
% \begin{document}
% \pptPic{0.2}{socrates.jpg} \\
% This is Socrates, a Greek philosopher.
% \end{document}
% \end{docshot}

% \DescribeMacro{\pptHeader}
% You can add a piece of code to the slide (we recommend using \href{https://github.com/yegor256/ffcode}{ffcode}):
% \begin{docshot}
% \documentclass{article}
% \usepackage[template,scheme=dark]{ppt-slides}
% \usepackage{ffcode}
% \begin{document}
% \pptHeader{This is How You Print to Console:}
% \begin{ffcode}
% void main(char** args) {
%   printf("Hello, world!");
% }
% \end{ffcode}
% \end{document}
% \end{docshot}
% Don't use |\plick| or |\plush|, they won't work with code snippets.
% The |\pptHeader| command prints a header similar to what the |\pptSection| command prints, but
% doesn't start a new section.

% \DescribeMacro{\pptQuote}
% \DescribeMacro{\pptBanner}
% You can put a quote next to the image, by using the |\pptQuote| command:
% \begin{docshot}
% \documentclass{article}
% \usepackage[template,scheme=dark]{ppt-slides}
% \begin{document}
% \pptBanner{My Favorite Quote}
% \pptQuote{socrates.jpg}{The only true wisdom
% is in knowing you know nothing}{Socrates}
% \end{document}
% \end{docshot}

% \DescribeMacro{\pptThought}
% For an interesting thought in the middle of the slide, use |\pptThought|:
% \begin{docshot}
% \documentclass{article}
% \usepackage[template,scheme=dark]{ppt-slides}
% \begin{document}
% \pptThought{Socrates said that the only true
% wisdom is in knowing you know nothing. Was he
% a wise man?...}
% \end{document}
% \end{docshot}

% \DescribeMacro{pptMiddle}
% You can put your content vertically in the middle of the page, using the |pptMiddle| environment:
% \begin{docshot}
% \documentclass{article}
% \usepackage[template,scheme=dark]{ppt-slides}
% \begin{document}
% \begin{pptMiddle}This is my text\end{pptMiddle}
% \end{document}
% \end{docshot}

% \DescribeMacro{\pptQR}
% Sometimes it's convenient to show a QR code to your audience instead of
% a URL, since it's easier to use --- they can scan it:
% \begin{docshot}
% \documentclass{article}
% \usepackage[template,scheme=dark]{ppt-slides}
% \begin{document}
% Check my blog post by this link: \\
% \pptQR{https://www.yegor256.com}
% \end{document}
% \end{docshot}
% The code will look like this, thanks to \href{https://ctan.org/pkg/qrcode}{qrcode} package that
% we use behind the scene.

% \DescribeMacro{pptWide}
% \DescribeMacro{pptWideOne}
% Sometimes you need your slide content to take all visible horizontal space:
% \begin{docshot}
% \documentclass{article}
% \usepackage[template,scheme=dark]{ppt-slides}
% \begin{document}
% \begin{pptWideOne}
% This paragraph is too long for some reasons,
% that's why must take all visible horizontal
% space, while normally the content has a wide
% margin on the left side of the page.
% \end{pptWideOne}
% \end{document}
% \end{docshot}
% You can also use |\begin{pptWide}{X}|,
% where |X| is the number of columns to render.

% \section{Package Options}

% \DescribeMacro{template}
% You start with a template for your slide deck:
%\iffalse
%<*verb>
%\fi
\begin{verbatim}
\documentclass{article}
\usepackage[template=9x6]{ppt-slides}
\begin{document}
...
\end{document}
\end{verbatim}
%\iffalse
%</verb>
%\fi
% There is only one template, which is used by default: |9x6|. If you don't
% specify the name, it will be used. If you don't use the |template| option at all,
% the default |article| will be rendered, which is not what you want.

% \DescribeMacro{scheme}
% You can choose a color scheme for your slides, using the |scheme| option
% of the package:
%\iffalse
%<*verb>
%\fi
\begin{verbatim}
\usepackage[template,scheme=light]{ppt-slides}
\end{verbatim}
%\iffalse
%</verb>
%\fi
% There are a few out-of-the-box schemes available: |light|, |dark|,
% |light-mono|, and |dark-mono|. You can design your own, using
% |ppt-light.tex| file as an example:
%\iffalse
%<*verb>
%\fi
\begin{verbatim}
\usepackage[template,scheme=/usr/local/my-colors.tex]{ppt-slides}
\end{verbatim}
%\iffalse
%</verb>
%\fi

% \DescribeMacro{directory}
% You can change the location of the templates and schemes by means of the |directory| package option:
%\iffalse
%<*verb>
%\fi
\begin{verbatim}
\usepackage[directory=/tmp,template=foo]{ppt-slides}
\end{verbatim}
%\iffalse
%</verb>
%\fi

% \DescribeMacro{nominutes}
% By default, if you render your slide deck in non-static mode (the |static| option for the ``clicks'' package),
% there will be minutes tracking in the right top corner of each slide. You can turn this off
% by using the |nominutes| option of the package:
%\iffalse
%<*verb>
%\fi
\begin{verbatim}
\usepackage[nominutes]{ppt-slides}
\end{verbatim}
%\iffalse
%</verb>
%\fi

% \StopEventually{}

% \section{Implementation}
% \changes{0.0.1}{2022/09/11}{First draft.}
% \changes{0.1.4}{2022/09/15}{The \texttt{nominutes} package option added.}
% \changes{0.1.5}{2022/09/19}{The default directory fixed.}
% \changes{0.2.0}{2022/12/03}{We migrated to the \texttt{ltxdoc} format and \texttt{.dtx} file.}

% First, we parse the package options:
%    \begin{macrocode}
\RequirePackage{pgfopts}
\makeatletter
\pgfkeys{
  /ppt-slides/.is family,
  /ppt-slides,
  template/.estore in = \ppt@template,
  template/.default = 9x6,
  scheme/.estore in = \ppt@scheme,
  directory/.estore in = \ppt@directory,
  directory/.default = ,
  nominutes/.estore in = \ppt@nominutes,
  directory
}
\ProcessPgfPackageOptions{/ppt-slides}
\makeatother
%    \end{macrocode}

% Then, we include the template:
%    \begin{macrocode}
\makeatletter\ifdefined\ppt@template
  \input{\ppt@directory ppt-templates/ppt-\ppt@template.tex}
\else
  \message{^^Jppt: No template is loaded, because
    the 'template' option is not specified}
\fi\makeatother
%    \end{macrocode}

% Then, we include the scheme:
%    \begin{macrocode}
\makeatletter\ifdefined\ppt@scheme
  \RequirePackage{xcolor}
  \input{\ppt@directory ppt-schemes/ppt-\ppt@scheme.tex}
\else
  \message{^^Jppt: No color scheme is loaded, because
    the 'scheme' option is not specified}
\fi\makeatother
%    \end{macrocode}

% Then, we include a few packages:
%    \begin{macrocode}
\RequirePackage{ifthen} % for if/then/else in LaTeX
\RequirePackage{href-ul} % for nice hyperlinks
\RequirePackage{pagecolor} % to change page color
\RequirePackage{varwidth} % for pptWide
\RequirePackage{qrcode} % to print QR codes
\RequirePackage{tikz} % for drawings in TOC
  \usetikzlibrary{calc} % for positioning of pinned boxes
\RequirePackage{tikzpagenodes} % for pptPin
\RequirePackage[inline]{enumitem} % for lists in crumbs
\RequirePackage{crumbs} % for crumbs
%    \end{macrocode}

% \begin{macro}{\pptMiddle}
% \changes{0.2.1}{2022/12/04}{The \texttt{\char`\\pptMiddle} command turned into \texttt{pptMiddle} environment.}
% Then, we define the |pptMiddle| environment:
%    \begin{macrocode}
\newenvironment{pptMiddle}
  {\vspace*{\fill}}
  {\vspace*{\fill}}
%    \end{macrocode}
% \end{macro}

% \begin{macro}{\pptBanner}
% Then, we define the |\pptBanner| command:
%    \begin{macrocode}
\newcommand\pptBanner[2][green]{%
  \par%
  {\setlength{\fboxsep}{6pt}%
    \colorbox{#1}{\color{\thepagecolor}\large #2}}
  \par%
}
%    \end{macrocode}
% \end{macro}

% \begin{macro}{\pptChapter}
% Then, we define the |\pptChapter| command:
%    \begin{macrocode}
\newcommand\pptChapter[2][]{%
  \crumbection[#1]{#2}%
  \begin{pptMiddle}%
    {\large\ttfamily Chapter \#\the\value{section}:}%
    \newline%
    \rotatebox{1}{\pptBanner[blue]{\LARGE#2}}%
  \end{pptMiddle}%
}
%    \end{macrocode}
% \end{macro}

% \begin{macro}{\pptSection}
% Then, we define the |\pptSection| command:
%    \begin{macrocode}
\newcommand\pptSection[2][]{%
  \subcrumbection[#1]{#2}%
  \pptHeader{#2}%
}
%    \end{macrocode}
% \end{macro}

% \begin{macro}{\pptHeader}
% Then, we define the |\pptHeader| command:
%    \begin{macrocode}
\newcommand\pptHeader[1]{%
  \vspace*{-0.75in}\hspace*{-0.5in}%
  \rotatebox{1}{\pptBanner[orange]{\Large #1}}\vspace*{0.15in}%
  \par%
}
%    \end{macrocode}
% \end{macro}

% \begin{macro}{\pptTitle}
% Then, we define the |\pptTitle| command:
%    \begin{macrocode}
\newcommand\pptTitle[2]{%
  {\pptBanner{\Huge #1}}%
  {\pptBanner{\large #2}}%
}
%    \end{macrocode}
% \end{macro}

% \begin{macro}{\pptToc}
% Then, we define the |\pptToc| command:
%    \begin{macrocode}
\newcommand*\pptTOC{}
\newcommand\pptToc[1][]{%
  \renewcommand*\pptTOC{}%
  \print{%
    \def\param{#1}%
    \ifx\param\empty\else%
      \pptBanner{#1}%
    \fi%
    \tikz{%
      \node (z) {%
        \begin{varwidth}{\textwidth}%
          \ifx\pptTOC\empty\else%
            \begin{itemize}[label={}]
              \pptTOC
            \end{itemize}
          \fi%
        \end{varwidth}%
      };%
      \path [draw=blue,line width=4pt] (z.north west) -- (z.south west);%
    }%
  }%
  \begin{@empty}%
    \renewcommand\crumb[2]{%
      \ifx\pptTOC\empty\else\ifdefined\click\click[0]\fi\fi
      \gappto\pptTOC{\item ##2}
    }
    \crumbs
  \end{@empty}%
  \ifdefined\flush\flush[1]\fi%
}
%    \end{macrocode}
% \end{macro}

% \begin{macro}{\pptQuote}
% Then, we define the |\pptQuote| command:
%    \begin{macrocode}
\RequirePackage{tabularx}
\newcommand\pptQuote[3]{%
  \begin{tabularx}{\columnwidth}{c>{\raggedright\arraybackslash}X}%
    \raisebox{\dimexpr-\height+\ht\strutbox}{\pptPic{0.25}{#1}}%
    &%
    ``#2''%
    \def\param{#3}%
    \ifx\param\empty\else%
      \newline\newline%
      \small%
      --- #3%
    \fi%
    \\%
  \end{tabularx}%
}
%    \end{macrocode}
% \end{macro}

% \begin{macro}{\pptPic}
% Then, we define the |\pptPic| command:
%    \begin{macrocode}
\newcommand\pptPic[2]{%
  {%
    \setlength{\fboxsep}{0pt}%
    \setlength{\fboxrule}{1pt}%
    \fcolorbox{gray}{white}{%
      \includegraphics[width=#1\columnwidth]{#2}%
    }%
  }%
}
%    \end{macrocode}
% \end{macro}

% \begin{macro}{\pptPin}
% Then, we define the |\pptPin| command:
%    \begin{macrocode}
\newcommand\pptPin[2][right]{%
  \begin{tikzpicture}[remember picture,overlay]%
    \ifthenelse{\equal{#1}{left}}%
    {%
      \node[anchor=north east] at
        ($(current page text area.north west) - (24pt, 18pt)$) {%
        \begin{minipage}{0.25\textwidth}%
          \raggedleft #2%
        \end{minipage}%
      };%
    }%
    {%
      \node[anchor=north east] at (current page text area.north east) {%
        \begin{minipage}{0.25\textwidth}%
          \raggedright #2%
        \end{minipage}%
      };%
    }%
  \end{tikzpicture}%
}
%    \end{macrocode}
% \end{macro}

% \begin{macro}{\pptThought}
% Then, we define the |\pptThought| command:
%    \begin{macrocode}
\newcommand\pptThought[1]{%
  \begin{pptMiddle}%
    \tikz{%
      \node [inner sep=18pt] (z) {%
        \begin{varwidth}{0.8\textwidth}
          \Large
          #1
        \end{varwidth}
      };
      \path [draw=green,line width=8pt] (z.north west) -- (z.south west);
    }%
  \end{pptMiddle}%
}
%    \end{macrocode}
% \end{macro}

% \begin{macro}{\pptSnippet}
% Then, we define the |\pptSnippet| command:
%    \begin{macrocode}
\newcommand\pptSnippet[2][\small]{%
  \begin{samepage}%
    #1\verbatiminput{#2}%
  \end{samepage}%
}
%    \end{macrocode}
% \end{macro}

% \begin{macro}{\pptQR}
% Then, we define the |\pptQR| command:
%    \begin{macrocode}
\newcommand\pptQR[2][2in]{%
  \tikz{%
    \node[draw=white]{%
      \href{#2}{%
        \qrcode[height=#1]{#2}%
      }%
    }%
  }%
}
%    \end{macrocode}
% \end{macro}

% \begin{macro}{\pptPinQR}
% Then, we define the |\pptPinQR| command:
%    \begin{macrocode}
\RequirePackage{seqsplit}
\newcommand\pptPinQR[2][right]{%
  \pptPin[#1]{%
    \begin{minipage}{1.4in}%
      \raggedleft%
      \pptQR[1in]{#2}%
      \\[3pt]
      \ttfamily\fontsize{8pt}{10pt}\selectfont%
      \seqsplit{#2} \href{#2}{$\rightarrow$}%
      \\
    \end{minipage}%
  }%
}
%    \end{macrocode}
% \end{macro}

% \begin{macro}{pptWide}
% Finally, we define the |pptWide| and |pptWideOne| environments:
%    \begin{macrocode}
\newenvironment{pptWideOne}
  {\begin{adjustwidth}{-2in}{-1in}}
  {\end{adjustwidth}}
\newenvironment{pptWide}[1]
  {\begin{pptWideOne}\begin{multicols}{#1}}
  {\end{multicols}\end{pptWideOne}}
%    \end{macrocode}
% \end{macro}

\endinput

% \Finale

% \clearpage
% \printbibliography
% \clearpage

% \PrintChanges
% \clearpage
% \PrintIndex
